\documentclass{article}
% maximum one page
\begin{document}
The set of polyhedral pointed rational cones form a partially ordered set with respect to elementary additive extensions of certain type. This poset captures a global picture of the interaction of all possible rational cones with the integer lattice and, also, provides
an alternative approach to another important poset, the poset of normal polytopes. One of the central conjectures in the field is the so called Cone Conjecture: the order on cones is the inclusion order. The conjecture has been proved only in dimensions up to 3. In this
work we develop an algorithmic approach to the conjecture in higher dimensions. Namely, we study how often two specific types of cone extensions generate the chains of cones in dimensions 4 and 5, whose existence follows from the Cone Conjecture. A na\"ive expectation, explicitly expressed in a recently published paper by Dr. Gubeldaze, is that these special extensions suffice to generate the desired chains. This would prove the conjecture in general and was the basis of the proof of the 3-dimesional case. Our extensive computational experiments show that in many cases the desired chains are in fact generated, but there are cases when the chain generation process does not terminate in reasonable time. Moreover, the fast generation of the desired chains fails in an interesting way---the complexity of the involved cones, measured by the size of their Hilbert bases, grows roughly linearly in time, making it less and less likely that we have a terminating process. This phenomenon is not observed in dimension 3. Our computations can be done in arbitrary high dimensions. We make a heavy use of SAGE, an open-source mathematics software system, and Normaliz, a C++ package designed to compute the Hilbert bases of cones.
\end{document}