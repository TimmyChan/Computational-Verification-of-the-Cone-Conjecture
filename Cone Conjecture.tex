
%--------------------------------------------%
% Packages arranged by : Tsz Timmy Chan	     %
%                 Date : November 26th, 2016%
%--------------------------------------------%

\documentclass{TC}
\usepackage{TCcommon}

\title{TITLE HERE}	% Work Title Here.
\author{Tsz Timmy Chan}	% YOUR NAME HERE 

\usepackage[notes]{TCheader}
\usepackage{TCexamtitle}
%\renewcommand{\benediction}{" " - }
%\renewcommand{\quoteoftheday}{" " \\ - }
\begin{document}
%\researchnote{Introduce the cone conjecture, then reference the proof that it's true in dimension three - by other connectivity results in arbitrary dimensions in the same paper, there is hope that this conjecture is true in any dimension.Describe the "special moves" (top down, bottom up) - used in proof for dimension 3 (aim to write this in mathematical language, and NEED TO PROVE that they are stonger than the cone conjecture).The paper explicitly asks if these elementary moves can generate the inclusion order. (reference this)}

The cone conjecture in \cite{GubeladzePosetCones} states that for every $d$, the order in $\mathrm{Cones}(d)$ is the inclusion order.

This conjecture is already proven in dimension $\leq 3$. For dimension 4, Andreas Paffenholz had implemented height-1 extensions and Hilbert basis  descends based on many randomly generated cones $C$ and vector $v$, with $\pm v \notin C$, which support the hypothesis that there are no non-terminating processes of either type.

The research presented in this thesis aims to refine this computational experimentation, and to generalize the implementations to any $C \subset D \in \R^d$ with any $d \geq 2$, in particular with a focus on dimension 5. 

\end{document}