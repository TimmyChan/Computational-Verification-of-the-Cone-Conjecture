
%--------------------------------------------%
% Packages arranged by : Tsz Timmy Chan	     %
%                 Date : November 26th, 2016%
%--------------------------------------------%

\documentclass{TC}
\usepackage{TCcommon}

\title{TITLE HERE}	% Work Title Here.
\author{Tsz Timmy Chan}	% YOUR NAME HERE 

\usepackage[notes]{TCheader}
\usepackage{TCexamtitle}
%\renewcommand{\benediction}{" " - }
%\renewcommand{\quoteoftheday}{" " \\ - }
\begin{document}
$\mathrm{Cones}(d)$ contains many elementary extensions of two different types, which makes this poset essentially different from $\NPol(d-1)$.

Let $C \subset \R^d$ be a full dimensional cone and have $v \in \Z^d$ on the exterior of $C$ with $\pm v \notin C$. We then denote the set $\F^+(v)$ as the set of facets visible from $v$. Each support hyperplane of the facets of $C$ has a linear form $ht_F(v) $ where $ht_F(C) \geq 0$. From the perspective of any point that is on the "other side" of this support hyperplane ($ht_F(v) < 0$) would consider this facet visible. 

We collect the visible parts of the boundary $\partial C$, and we'll label this set $$C^+(v) = \displaystyle\bigcup_{F \in \F^+(v)} F,$$ i.e., take the union of all the facets visible to $v$. We then take the conical hull of $C$ and $v$ to form a new cone $D = C + \R_+ v$. By this construction, there exists an increasing sequence of rational numbers $$0 < \lambda_1 < \lambda_2 < \cdots$$ such that $$\lambda_1 = \frac{1}{\max(-\mathrm{ht}_F(v))} \;\; \text{ for } F \in \F^+(v) $$ and $\lim_{k \to \infty} \lambda_k = \infty$, satisfying the following  conditions:
\begin{align*}
\Lattice(D \takeaway C) &= \bigcup_{k=1}^\infty \Lattice(\lambda_k v + C^+(v)) \\
\Lattice(\lambda_k v + C^+(v)) &\neq \emptyset \;\; \text{ for } k = 1,2,\dots
\end{align*}
When $\lambda_1 = 1$, or equivalently, $\mathrm{ht}_F(v) = -1$ for all $F \in \F^+(v)$. In this special case, we say $D$ is a \emph{height 1 extension} of $C$. While all height 1 extensions are elementary extensions, the converse is not true \cite{BrunsGubeladzeNormalPolytopes}. 


\begin{lemma}\label{bottomupposetlemma}
The \emph{Bottom Up} algorithm is an implementation of height-1 extensions, which generate cones $C = C_0 \subset C_1 \subset C_2 \subset \cdots$ such that each pair $C_{n} \subset C_{n+1}$ satisfy the poset condition \textbf{\ref{ConePosetCondition}}.
\end{lemma}

\end{document}