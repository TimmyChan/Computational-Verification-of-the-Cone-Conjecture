
%--------------------------------------------%
% Packages arranged by : Tsz Timmy Chan	     %
%                 Date : November 26th, 2016%
%--------------------------------------------%

\documentclass{TC}
\usepackage{TCcommon}

\title{TITLE HERE}	% Work Title Here.
\author{Tsz Timmy Chan}	% YOUR NAME HERE 

\usepackage[notes]{TCheader}
\usepackage{TCexamtitle}
%\renewcommand{\benediction}{" " - }
%\renewcommand{\quoteoftheday}{" " \\ - }
\begin{document}
Included here are code that is written in \texttt{Python} for this research, and they are also concurrently hosted on 


\begin{center}
\begin{minipage}{.75\textwidth}
\begin{mdframed}
\texttt{http://www.github.com/TimmyChan/ConeThesis}
\end{mdframed}
\end{minipage}
\end{center}

\texttt{cone\_tools.py} This file contains mathematical functions written for the project:
	\begin{itemize}
		\item \texttt{cone\_containment(C,D)}: Verifies C is contained in D
		\item \texttt{shortest\_vector(vectorlist), longest\_vector(vectorlist)}: Returns the shortest and longest vectors from a list, respectively.
		\item \texttt{make\_primitive(vectlist)}: Returns a primitive vector along the ray generated by some vector in $\Z^d$.
		\item \texttt{generate\_random\_vector(dim, rmax=10)}: Generates a random vector in $v \in \Z^d$ such that $\abs{v_i} \leq d$ and $v_d \geq 1$. The last condition is used to guarantee the following function always generates a cone contained in the halfspace $x_d > 0$, so it contains no lines. Using the previous function we always generate a random \emph{primitive} vector.
		
		\item \texttt{generate\_cone(dim, rmax=10,numgen=10), generate\_inner\_cone(outer)}: Generates a cone using random vectors constrained by the dimension (dim) and the random generation bounds. Numgen is the number of vectors used to generate the cone. The latter generates a cone that is guaranteed to be contained by the outer cone.
		
		\item \texttt{extremal\_generators\_outside\_inner\_cone(inner, outer)}: This is used to find a list of extremal generators at the beginning of the Top Down algorithm.
		
		\item \texttt{visible\_facets(cone,vect), facets\_with\_max\_lambda(visiblefacets,v)}: These are used in the Bottom Up algorithm.
		
		\item \texttt{vector\_sum(listofvectors), zonotope\_generators(vectlist)}: Generating a zonotope in $\Z^d$ using the power set of a set of vectors, then taking sum of each element of the power set, and adjoining the empty set as the origin.
				
	\end{itemize}

\newpage
\section{cone\_tools.py}
\input{"cone_tools.py.tex"}
\newpage
\section{cone\_chain\_element.py}
\input{"cone_chain_element.py.tex"}
\newpage
\section{cone\_chain.py}
\input{"cone_chain.py.tex"}
\newpage
\section{cone\_conjecture\_tester.py}
\input{"cone_conjecture_tester.py.tex"}
\end{document}