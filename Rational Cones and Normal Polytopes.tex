
%--------------------------------------------%
% Packages arranged by : Tsz Timmy Chan	     %
%                 Date : November 26th, 2016%
%--------------------------------------------%

\documentclass{TC}
\usepackage{TCcommon}

\title{TITLE HERE}	% Work Title Here.
\author{Tsz Timmy Chan}	% YOUR NAME HERE 

\usepackage[notes]{TCheader}
\usepackage{TCexamtitle}
%\renewcommand{\benediction}{" " - }
%\renewcommand{\quoteoftheday}{" " \\ - }
\begin{document}


The partial order on the set of normal polytopes is preserved; i.e., the homogenization map (equations \ref{H-homogenization_map} and \ref{V-homogenization_map}, which are non-trivially equivalent) is monotonic with respect to the partial order. This means that  $P \subset Q \in \mathrm{NPol}(d-1)$, then $C(P) \subset C(Q) \in \mathrm{Cones}^+(d)$ and $(P < Q) \implies C(P) < C(Q)$. However, the converse is not necessarily true. 
In the following example, found in \cite{BrunsGubeladzeNormalPolytopes}:
\begin{example}
Suppose $$ P = \mathrm{conv}(\{(0,0,2),(0,0,1),(0,1,3),(1,0,0),(2,1,2),(1,2,1)\}) \in \NPol(3).$$
Then removing the first or the second vertex yields a non-normal polytope, yet $$Q = \mathrm{conv}(\{0,1,3),(1,0,0),(2,1,2),(1,2,1)\}) \subset P \in \NPol(3). $$
Thus, this pair of normal polytopes $P \subset Q$ in $\NPol(3)$ does not satisfy the poset condition, i.e. $Q \not < P$, but when considering their images in $\mathrm{Cone}^+(d)$, namely, $C(Q) \subset C(P)$, one can find a finite chain that that connects the two in the poset of cones:
\begin{align*}
C(P) &= \mathrm{cone}\{(0, 0, 1, 1), (0, 0, 2, 1), (0, 1, 3, 1), (1, 0, 0, 1), (1, 2, 1, 1), (2, 1, 2, 1)\})\\
C(Q) &= \mathrm{cone}(\{(0, 1, 3, 1), (1, 0, 0, 1), (1, 2, 1, 1), (2, 1, 2, 1)\}).
\end{align*}
Then
\begin{align*}
 C(P) = D_0 &=  \mathrm{cone}(\{(0, 0, 1, 1),(0, 0, 2, 1),(0, 1, 3, 1),(1, 0, 0, 1),(2, 1, 2, 1),(1, 2, 1, 1)\}) \\
 D_1 &= \mathrm{cone}(\{(0, 0, 2, 1), (0, 1, 3, 1), (1, 0, 0, 1), (1, 2, 1, 1), (2, 1, 2, 1)\})\\
 D_2 &= \mathrm{cone}(\{(0, 1, 3, 1),(1, 0, 0, 1),(2, 1, 3, 2),(2, 1, 2, 1),(1, 2, 1, 1)\})\\
 D_3 &= \mathrm{cone}(\{(0, 1, 3, 1),(1, 0, 0, 1),(2, 1, 2, 1),(1, 2, 1, 1)\}) = C(Q).
 \end{align*}
 The above was found using the Hilbert descend, or the "top down" method. Note that this required Hilbert basis elements at height two; so the polytopes formed by $\left\{ x \in \R^d: \begin{bmatrix} 1 \\ x\end{bmatrix} \in C_i \right\}$ are not normal.
\end{example}




\end{document}